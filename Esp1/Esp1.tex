\documentclass[11pt,a4paper]{article}
\usepackage[utf8]{inputenc}
\usepackage[italian]{babel}
\usepackage{amsmath}
\usepackage{amsfonts}
\usepackage{amssymb}
\usepackage{graphicx}
\author{Volpini, Paganini, Finazzer, Beghini}
\title{Energia di Ionizzazione }
\begin{document}
\maketitle 
\section{Abstract}
Nella prima parte di questa esperienza si vuole studiare il fenomeno di produzione di correnti elettroniche per effetto termoionico in una camera da vuoto; in particolare si verificano sperimentalmente le leggi di Richardson e di Child. Nella seconda parte si studia invece il processo di ionizzazione di gas residui nella camera da vuoto; in particolare sperimentalmente si vogliono misurare le diverse energie di ionizzazione di alcune specie di gas.  
\section{Apparato sperimentale}
Come si evince dalla figura le principali componenti dell'apparato sperimentale sono:
\begin{itemize}
\item camera da vuoto, 
\item misuratori di pressione Pennig (catodo freddo) e Pirani
\item pompa rotativa e pompa turbo-molecolare
\item valvola a spillo, valvola gate
\item flange di varie dimensioni
\item cavi banana-banana
\item valvola termoionica artigianale con filamento di tungsteno
\item generatori di corrente (Agilent e boh), multimetri digitali e palmari e un elettrometro
\end{itemize} 


\section{Prima parte}
Si comincia generando il vuoto all'interno della camera con le dovute accortezze. Si utilizzano pompa rotativa e pompa turbomolecolare per raggiungere condizioni di medio-alto vuoto ({P=10-6 mbar}.

instauriamo una differenza di potenziale ai capi del filamento collegandone gli estremi rispettivamente uno a massa ed uno ad un generatore (per motivi sperimentali si è scelto quello in grado di erogare maggiori correnti); in questo modo è possibile riscaldare il filamento fino al raggiungimento di temperature tali da provocarne l'effetto termoionico ({T>1800 K}). Il secondo generatore si usa invece per generare una differenza di potenziale tra filamento e anodo, e catturare quindi gli elettroni che si liberano dal filamento. Per la prima parte dell'esperienza il collettore rimane scollegato dal sistema in quanto non necessario.

\subsection{dati sperimentali}

si comincia con alcune misure specifiche del filamento del tungsteno utili a calcolare Resistenza e resistività dello stesso:
\begin{itemize}
\item diametro filamento ${d=205+-5\mu}$ 
\item lunghezza filamento ${l=20.70+-0.05mm}$
\item resistena (4 wire) filamento campione + fili ${}$
\item lunghezza filamento campione ${}$
\end{itemize}


Utilizziamo il filamento campione per calcolare la resistività del filamento utilizzato nella valvola termoionica (è una proprietà del materiale non del singolo filamento) $\rho=R_{s}\pi (d/2)^{2})/l_{s}=$ (qui dobbiamo utilizzare quelle citate sopra).  non ci siamo fidati del valore presentato nella letteratura scientifica in quanto la purezza del filamento in questione è sconosciuta (abbiamo comunque confrontato i due valori per quanto riguarda l'ordine di grandezza).

A questo punto è possibile calcolare la resistenza del filamento inserito nella valvola e soggetto ad effetto termoionico:
$ r_{fil}=l\rho/\pi (d/2)^{2})=$. 
dobbiamo a questo punto considerare il contributo alla resistenza dato dai cavi banana-banana utilizzati per il circuito in esame:\\
calcoliamo la resistenza dei cavi mediante la relazione $r_{cavi}=r_{tot}-r_{fil}=$(modalità "4 wire" del multimetro). Supponendo che tali cavi non si riscaldino durante l'esperienza siamo in grado di separare i diversi contributi nel calcolo della resistenza $r_{tot}=r_{cavi}+r_{fil}$.

successivamente si procede ad una stima della temperatura raggiunta dal filamento sottoposto ad una certa differenza di potenziale.
La letteratura scientifica fornisce il seguente datasheet: (metterei il datasheet)

Da tale grafico si ricava la seguente relazione $\dfrac{r_{fil}}{r_{T300k}}=\alpha \Delta T $. Invertendo la relazione stimiamo la temperatura finale del filamento (dobbiamo mettere dei valori di temperatura: io metterei i due soglia).

Siamo ora interessati a verificare la legge di richardon che studia l'andamento della corrente degli elettroni prodotti per effetto termoionico in funzione della temperatura del filamento (metterei la formula).
Impostiamo a questo punto alcuni parametri costanti (${P=}$, ${Va=}$, ) e procediamo a misurare valori di corrente elettronca al variare del potenziale di filamento (Vfil) e quindi della temperatura dello stesso (metterei i calcoli per passare da Vfil a Tfil).

si riassumono i risultati nel seguente grafico

(mettere grafico e commentare)

Si studia infine l'andamento della corrente elettronica in funzione del potenziale dell'anodo. a questo scopo è necessario procedere con più serie effettuate a diverso Vfil e di conseguenza a diversa Tfil.
si verifica che nel range a bassi potenziali la legge di Child fitti sufficientemente bene i punti presi sperimentalmente (mettere formula child)

si riassumono i risultati nel seguente grafico

(mettere grafico e commentare:-prima parte fittata da child e nella seconda parte corrente saturata)


conclusioni prima parte:
-i fit vengono bene?
-cosa abbiamo capito sulla corrente elettronica prodotta da effetto termoionico?



\section{Seconda Parte}
 
si procede in questa seconda parte dell'esperienza a studiare il fenomeno di ionizzazione di alcune specie di gas. é necessario a questo punto collegare opportunamente anche il collettore della valvola termoionica; in questo modo raccoglieremo gli atomi ionizzati sulla griglia esterna e ne potremmo misurare la corrente ionica. accertandosi inoltre di tenere il collettore ad un potenziale inferiore rispetto al filamento si garantisce una barriera di potenziale per gli elettroni prodotti per effetto termoionico che rimangono tra filamento e anodo e continuano a ionizzare i gas presenti in camera.

\subsection{dati sperimentali}

in primo luogo verifichiamo che all'aumentare della pressione nella camera da vuoto aumenti anche la corrente ionica misurata (per fare cio si mantengono costanti i parametri quali? e si apre la valvola a spillo per variare i valori di pressione in camera); questo è dovuto al fatto che aumentando la pressione in camera aumentiamo il numero di atomi della specie di gas e aumentiamo quindi il numero di atomi che possono essere ionizzati dagli elettroni.

si riassume quanto spiegato sopra nel seguente grafico:

(mettere grafico e commentare)

successivamente si procede studiando l'andamento della corrente ionica in funzione del potenziale di anodo. il grafico contenente i dati sperimentali per due diverse specie di gas ci permette mediante tecniche di fit di stimare le diverse energie di ionizzazione.
si sottolinea che aumentando il potenziale di anodo aumenta anche la corrente elettronica prodotta per effetto termoionico; si provvede quindi ad effettuare una normalizzazione che permetta di considerare solo il contributo dell'energia degli elettroni bombardati e non del loro numero. (sull'asse delle ordinate del grafico si riporta quindi il rapporto Ii/Ie)

si riassumono i risultati nel seguente grafico

(mettere grafico e commentare)

conclusioni seconda parte
siamo riusciti a calcolare le energie di ionizzazione, sono sensate? sono sufficientemente diverse?
perchè la curva sale molto dolcemente invece che avere un comportamento a gomito? (discorso distribuzione di velocità dovute alla statistica ma soprattutto al fatto che il filamento è soggetto ad una differenza di potenziale) 
ci aspettiamo che raggiunga un massimo e poi cala! perchè? perchè non lo vediamo? (non potevamo aumentare il potenziale e quindi non lo raggiungiamo neanche)

\subsection{conclusioni}
metterei delle conclusioni generali che riassumono tutta l'esperienza (praticamente l'abstract)

\end{document}